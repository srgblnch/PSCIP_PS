\hypertarget{pscip_8c}{
\section{Z:/Project/development/PSCIP/driver/pscip.c File Reference}
\label{pscip_8c}\index{Z:/Project/development/PSCIP/driver/pscip.c@{Z:/Project/development/PSCIP/driver/pscip.c}}
}
PSCIP Driver for Linux Kernel 2.6.x-development version - main file.  


{\tt \#include $<$linux/init.h$>$}\par
{\tt \#include $<$linux/module.h$>$}\par
{\tt \#include $<$linux/moduleparam.h$>$}\par
{\tt \#include $<$linux/fs.h$>$}\par
{\tt \#include $<$linux/types.h$>$}\par
{\tt \#include $<$linux/kdev\_\-t.h$>$}\par
{\tt \#include $<$linux/cdev.h$>$}\par
{\tt \#include $<$linux/kernel.h$>$}\par
{\tt \#include $<$linux/pci.h$>$}\par
{\tt \#include $<$linux/ioport.h$>$}\par
{\tt \#include $<$linux/interrupt.h$>$}\par
{\tt \#include $<$asm/spinlock.h$>$}\par
{\tt \#include $<$asm/semaphore.h$>$}\par
{\tt \#include $<$asm/system.h$>$}\par
{\tt \#include $<$asm/io.h$>$}\par
{\tt \#include $<$linux/proc\_\-fs.h$>$}\par
{\tt \#include $<$linux/errno.h$>$}\par
{\tt \#include $<$linux/delay.h$>$}\par
{\tt \#include $<$linux/time.h$>$}\par
{\tt \#include $<$asm/uaccess.h$>$}\par
{\tt \#include $<$asm/atomic.h$>$}\par
{\tt \#include \char`\"{}debug\_\-pscip.h\char`\"{}}\par
{\tt \#include \char`\"{}pscip.h\char`\"{}}\par
{\tt \#include \char`\"{}k\_\-pscip.h\char`\"{}}\par
{\tt \#include \char`\"{}k\_\-pscip.c\char`\"{}}\par
{\tt \#include \char`\"{}debug\_\-pscip.c\char`\"{}}\par


Include dependency graph for pscip.c:\subsection*{Defines}
\begin{CompactItemize}
\item 
\#define \hyperlink{pscip_8c_928bcd4ed1ca26fa864fe5516efa2b20}{\_\-\_\-DEBUG\_\-\_\-}
\begin{CompactList}\small\item\em If defined, enables debug printk to logfile. \item\end{CompactList}\end{CompactItemize}
\subsection*{Functions}
\begin{CompactItemize}
\item 
\hyperlink{pscip_8c_d94b36675e7eb067ea3ce6ff9e244a44}{MODULE\_\-LICENSE} (\char`\"{}GPL\char`\"{})
\begin{CompactList}\small\item\em $<$ If defined, Real Time IP enabled. \item\end{CompactList}\item 
\hyperlink{pscip_8c_eeb0a2d9412b2598de60d3ab78abd559}{module\_\-param} (\hyperlink{pscip_8c_1aa2a01bfd35c87921617d64a164bda7}{hack\_\-driver}, int, 0)
\begin{CompactList}\small\item\em Macro defining hack\_\-driver module parameters. \item\end{CompactList}\item 
\hyperlink{pscip_8c_b83342c5da927e07b71f9e9e0f2ac1ba}{module\_\-param} (\hyperlink{pscip_8c_c42777cda72b563c2676c06d0fd98c1a}{pscip\_\-major}, int, 0)
\begin{CompactList}\small\item\em Macro defining pscip\_\-major module parameters. \item\end{CompactList}\item 
\hyperlink{pscip_8c_486932d17bb9779ad2d987b610b1ab88}{MODULE\_\-PARM\_\-DESC} (\hyperlink{pscip_8c_c42777cda72b563c2676c06d0fd98c1a}{pscip\_\-major},\char`\"{}PSCIP major number, if defined by user\char`\"{})
\item 
\hyperlink{pscip_8c_4c7ea419630b353c78ac1aaf9cdd3a78}{MODULE\_\-PARM\_\-DESC} (\hyperlink{pscip_8c_5227c6d12c17c33f868af17fb077dc62}{pscip\_\-dev\_\-number},\char`\"{}PSCIP number of devices which can be run using this driver\char`\"{})
\item 
\hyperlink{pscip_8c_905ed87697ca0a454ff57d69ec22af48}{MODULE\_\-PARM\_\-DESC} (\hyperlink{pscip_8c_1aa2a01bfd35c87921617d64a164bda7}{hack\_\-driver},\char`\"{}PSCIP special functions for developers:$\backslash$n -- no func implemented at the moment --\char`\"{})
\item 
void \hyperlink{pscip_8c_fbfd4fc2195d42c9aaed821f40c33e12}{pscip\_\-irq\_\-bh0} (unsigned long)
\begin{CompactList}\small\item\em Tasklet used to send a SIGIO to user space, chan 0. \item\end{CompactList}\item 
void \hyperlink{pscip_8c_0792694f70567c97210fbbee0e0702a3}{pscip\_\-irq\_\-bh1} (unsigned long)
\begin{CompactList}\small\item\em Tasklet used to send a SIGIO to user space, chan 0. \item\end{CompactList}\item 
int \hyperlink{pscip_8c_893e661d3b98b489ec148f53552157ef}{pscip\_\-assign\_\-irq} (\hyperlink{structnpscip___dev}{pscip\_\-Dev} dev)
\begin{CompactList}\small\item\em Function registering interrupt in the cernel. \item\end{CompactList}\item 
irqreturn\_\-t \hyperlink{pscip_8c_0356476f23b4337ef94d580026be8bbb}{pscip\_\-irq\_\-handler} (int pciirq, void $\ast$dev\_\-id, struct pt\_\-regs $\ast$regs)
\begin{CompactList}\small\item\em Interrupt hanlder. \item\end{CompactList}\item 
static int \hyperlink{pscip_8c_38ed0a47b49920ca3b02392764122bed}{pscip\_\-open} (struct inode $\ast$, struct file $\ast$)
\begin{CompactList}\small\item\em Function called when the node represeting driver is opened. \item\end{CompactList}\item 
static int \hyperlink{pscip_8c_ac05b05bbeed23bd53e662d75d14d4f8}{pscip\_\-release} (struct inode $\ast$, struct file $\ast$)
\begin{CompactList}\small\item\em Function called when the node represeting driver is closed. \item\end{CompactList}\item 
static int \hyperlink{pscip_8c_189aa0b40518a5c712bb9c24f4da5e7f}{pscip\_\-ioctl} (struct inode $\ast$, struct file $\ast$, unsigned int, unsigned long)
\begin{CompactList}\small\item\em Function implementing ioctl control. \item\end{CompactList}\item 
static int \hyperlink{pscip_8c_1f2bd07aac6fb1d9a27f635dac28669b}{probe} (struct pci\_\-dev $\ast$dev, const struct pci\_\-device\_\-id $\ast$id)
\begin{CompactList}\small\item\em Function called when the device (Carrier card) is detected on the PCI bus. \item\end{CompactList}\item 
static void \hyperlink{pscip_8c_fdbceb4ede4f982e2e68f2ee7c0cf6e1}{remove} (struct pci\_\-dev $\ast$pcidev)
\begin{CompactList}\small\item\em Function called when the device (Carrier) is removed from the PCI bus. \item\end{CompactList}\item 
static int \hyperlink{pscip_8c_8675bb66851e6553e8d99ec533e2f6ec}{pscip\_\-read\_\-proc} (char $\ast$page, char $\ast$$\ast$start, off\_\-t offset, int count, int $\ast$eof, void $\ast$data)
\begin{CompactList}\small\item\em Create a proc filesystem read entry. \item\end{CompactList}\item 
static int \hyperlink{pscip_8c_bf1022c0b7beb0fa86c449b1ba63d58b}{pscip\_\-write\_\-proc} (struct file $\ast$filp, const char $\ast$buffer, unsigned long count, void $\ast$data)
\begin{CompactList}\small\item\em creat a proc file system write entry \item\end{CompactList}\item 
static int \hyperlink{pscip_8c_4c7cd65a5fb7a2c6dc317b41df5622dd}{pscip\_\-procfs\_\-register} (void)
\begin{CompactList}\small\item\em Register proc filesystem entries. \item\end{CompactList}\item 
static void \hyperlink{pscip_8c_d74a62d23d8b05733bfe84136d4c3431}{init\_\-pscip} (int i)
\begin{CompactList}\small\item\em Initializatin of interrupt and concurrency related mechanisms. \item\end{CompactList}\item 
static int \hyperlink{pscip_8c_6a25efed588d9784c4b9e53ae3c7a5b8}{checkmem\_\-device} (int i)
\begin{CompactList}\small\item\em Checks whether device is in the slot by reading specific registers holdilng IP IDs. \item\end{CompactList}\item 
static void \hyperlink{pscip_8c_a07850d4e960afda29741908877b9f72}{exit\_\-pscip\_\-module} (void)
\begin{CompactList}\small\item\em Function called when the module is deinserted. \item\end{CompactList}\item 
static int \hyperlink{pscip_8c_f6ae932073296558700ee54b8c3ba8b2}{init\_\-pscip\_\-module} (void)
\begin{CompactList}\small\item\em Function called when the driver/module is inserted. \item\end{CompactList}\item 
\hyperlink{pscip_8c_a5f2885f1cd67ce96d2afa5a402033a8}{module\_\-init} (init\_\-pscip\_\-module)
\begin{CompactList}\small\item\em Registering first function to be called when inserting the module. \item\end{CompactList}\item 
\hyperlink{pscip_8c_cc378f45db0de3c2cc0ac3b485843a26}{module\_\-exit} (exit\_\-pscip\_\-module)
\begin{CompactList}\small\item\em Registering last function to be called when de-inserting the module. \item\end{CompactList}\end{CompactItemize}
\subsection*{Variables}
\begin{CompactItemize}
\item 
static int \hyperlink{pscip_8c_8883057b44748f328f9ab36e5bcf2a8f}{queue\_\-flag} = 0
\begin{CompactList}\small\item\em Flag used in waitqueue mechanism. \item\end{CompactList}\item 
int \hyperlink{pscip_8c_422f87b12a1ce74ba2ba50eacb4ee12e}{pscip\_\-minor} = PSCIP\_\-MINOR\_\-NUMBER
\begin{CompactList}\small\item\em Global variable with first minor numer associated with the driver. \item\end{CompactList}\item 
static char \hyperlink{pscip_8c_699de8ad6de47a5f252fe943bbe146e5}{ver} \mbox{[}10\mbox{]} = \char`\"{}1.0\char`\"{}
\begin{CompactList}\small\item\em Module/driver version. \item\end{CompactList}\item 
const char $\ast$ \hyperlink{pscip_8c_4c2d9327de87c6be0b54f793adc7f1fb}{module\_\-name} = \char`\"{}pscip\char`\"{}
\begin{CompactList}\small\item\em module name. \item\end{CompactList}\item 
static int \hyperlink{pscip_8c_5227c6d12c17c33f868af17fb077dc62}{pscip\_\-dev\_\-number} = 0
\begin{CompactList}\small\item\em Module parameter - number of IPs. \item\end{CompactList}\item 
static int \hyperlink{pscip_8c_1aa2a01bfd35c87921617d64a164bda7}{hack\_\-driver} = 0
\begin{CompactList}\small\item\em Module paramter - options for developer. \item\end{CompactList}\item 
static int \hyperlink{pscip_8c_c42777cda72b563c2676c06d0fd98c1a}{pscip\_\-major}
\begin{CompactList}\small\item\em Global variable with major numer associated with the driver. \item\end{CompactList}\item 
static struct file\_\-operations \hyperlink{pscip_8c_1969c4e32e8a199df965d0e49e7bdb8e}{pscip\_\-fops}
\begin{CompactList}\small\item\em File operation structure. \item\end{CompactList}\item 
static struct pci\_\-driver \hyperlink{pscip_8c_b8e932f1cc0da35f7c08d4bcfb8afaee}{pscip\_\-driver}
\begin{CompactList}\small\item\em Structure describing PCI device (IP). \item\end{CompactList}\item 
static \hyperlink{structnpscip___proc}{pscip\_\-Proc} \hyperlink{pscip_8c_8f2d8a5dd9c85bb225fbd2edf4175095}{pscip\_\-procdev}
\begin{CompactList}\small\item\em Proc filesystem structure . \item\end{CompactList}\item 
static struct proc\_\-dir\_\-entry $\ast$ \hyperlink{pscip_8c_e88af066b7442e9fc6cf5228860c2cc5}{proc\_\-model\_\-dir}
\begin{CompactList}\small\item\em Structure defining entry in proc file system. \item\end{CompactList}\end{CompactItemize}


\subsection{Detailed Description}
PSCIP Driver for Linux Kernel 2.6.x-development version - main file. 

GG - 19.07.2005: First implementation

22/10/2008 - Start of porting the driver :\begin{itemize}
\item from 2.4.x to 2.6.x\item from VME to PCIGG 12/20008 - finalizing first release \end{itemize}


Definition in file \hyperlink{pscip_8c-source}{pscip.c}.

\subsection{Define Documentation}
\hypertarget{pscip_8c_928bcd4ed1ca26fa864fe5516efa2b20}{
\index{pscip.c@{pscip.c}!\_\-\_\-DEBUG\_\-\_\-@{\_\-\_\-DEBUG\_\-\_\-}}
\index{\_\-\_\-DEBUG\_\-\_\-@{\_\-\_\-DEBUG\_\-\_\-}!pscip.c@{pscip.c}}
\subsubsection[{\_\-\_\-DEBUG\_\-\_\-}]{\setlength{\rightskip}{0pt plus 5cm}\#define \_\-\_\-DEBUG\_\-\_\-}}
\label{pscip_8c_928bcd4ed1ca26fa864fe5516efa2b20}


If defined, enables debug printk to logfile. 

Can be very useful when trying to determine bug location Every function prints information when beginning and when finishing Functions whose code is in \hyperlink{pscip_8c}{pscip.c} file print informatin starting with DEBUG\_\-M and are not intended Fundtions whose code is in \hyperlink{k__pscip_8c}{k\_\-pscip.c} file print information starting with DEBUG\_\-K and are intended Funnctions whose code is in debug\_\-pscip. file print information starting with DEBUG and are even more inteded 

Definition at line 69 of file pscip.c.

Referenced by k\_\-pscip\_\-rdstatistics().

\subsection{Function Documentation}
\hypertarget{pscip_8c_6a25efed588d9784c4b9e53ae3c7a5b8}{
\index{pscip.c@{pscip.c}!checkmem\_\-device@{checkmem\_\-device}}
\index{checkmem\_\-device@{checkmem\_\-device}!pscip.c@{pscip.c}}
\subsubsection[{checkmem\_\-device}]{\setlength{\rightskip}{0pt plus 5cm}static int checkmem\_\-device (int {\em i})\hspace{0.3cm}{\tt  \mbox{[}static\mbox{]}}}}
\label{pscip_8c_6a25efed588d9784c4b9e53ae3c7a5b8}


Checks whether device is in the slot by reading specific registers holdilng IP IDs. 



Definition at line 1174 of file pscip.c.

References IP\_\-read\_\-byte(), pscip\_\-devices, PSCIP\_\-MANIFAC\_\-ID, PSCIP\_\-MANIFAC\_\-REG, PSCIP\_\-MODEL\_\-ID, and PSCIP\_\-MODEL\_\-REG.

Referenced by probe().\hypertarget{pscip_8c_a07850d4e960afda29741908877b9f72}{
\index{pscip.c@{pscip.c}!exit\_\-pscip\_\-module@{exit\_\-pscip\_\-module}}
\index{exit\_\-pscip\_\-module@{exit\_\-pscip\_\-module}!pscip.c@{pscip.c}}
\subsubsection[{exit\_\-pscip\_\-module}]{\setlength{\rightskip}{0pt plus 5cm}static void exit\_\-pscip\_\-module (void)\hspace{0.3cm}{\tt  \mbox{[}static\mbox{]}}}}
\label{pscip_8c_a07850d4e960afda29741908877b9f72}


Function called when the module is deinserted. 

Clean up module from kernel - all unregister staff 

Definition at line 1417 of file pscip.c.

References module\_\-name, proc\_\-model\_\-dir, pscip\_\-dev\_\-number, pscip\_\-driver, pscip\_\-major, and pscip\_\-minor.\hypertarget{pscip_8c_d74a62d23d8b05733bfe84136d4c3431}{
\index{pscip.c@{pscip.c}!init\_\-pscip@{init\_\-pscip}}
\index{init\_\-pscip@{init\_\-pscip}!pscip.c@{pscip.c}}
\subsubsection[{init\_\-pscip}]{\setlength{\rightskip}{0pt plus 5cm}static void init\_\-pscip (int {\em i})\hspace{0.3cm}{\tt  \mbox{[}static\mbox{]}}}}
\label{pscip_8c_d74a62d23d8b05733bfe84136d4c3431}


Initializatin of interrupt and concurrency related mechanisms. 

The functions registers tasklests for each channel on the calling IP, registers spinlocks and waitqueues.

\begin{Desc}
\item[Parameters:]
\begin{description}
\item[{\em i}]- number of device ( IP ) in the device structure \end{description}
\end{Desc}


Definition at line 1135 of file pscip.c.

References pscip\_\-devices, pscip\_\-irq\_\-bh0(), pscip\_\-irq\_\-bh1(), PSCIP\_\-LINKDOWN\_\-TOUT, and npscip\_\-Dev::tv.

Referenced by probe().\hypertarget{pscip_8c_f6ae932073296558700ee54b8c3ba8b2}{
\index{pscip.c@{pscip.c}!init\_\-pscip\_\-module@{init\_\-pscip\_\-module}}
\index{init\_\-pscip\_\-module@{init\_\-pscip\_\-module}!pscip.c@{pscip.c}}
\subsubsection[{init\_\-pscip\_\-module}]{\setlength{\rightskip}{0pt plus 5cm}static int init\_\-pscip\_\-module (void)\hspace{0.3cm}{\tt  \mbox{[}static\mbox{]}}}}
\label{pscip_8c_f6ae932073296558700ee54b8c3ba8b2}


Function called when the driver/module is inserted. 

It establishes how many devices ( IPs ) the driver will enable to operate. It registers major and minor numbers for the driver, registers driver, prepare memory for device structure, initializes device structure table with 0/null values and create proc\_\-fs entries in /proc/driver folder. 

Definition at line 1455 of file pscip.c.

References npscip\_\-Dev::carrier\_\-number, FALSE, npscip\_\-Dev::IP\_\-in\_\-slot, npscip\_\-Dev::IP\_\-IOspaceAddr, npscip\_\-Dev::IP\_\-IOspaceFlags, npscip\_\-Dev::IP\_\-IOspaceLen, npscip\_\-Dev::IPinterfaceAddr, npscip\_\-Dev::IPinterfaceFlags, npscip\_\-Dev::IPinterfaceLen, npscip\_\-Dev::IPnumber, npscip\_\-Dev::irq, npscip\_\-Dev::irqdata, npscip\_\-Dev::irqflag, npscip\_\-Dev::pciaddr, npscip\_\-Dev::PCIinterfaceAddr, npscip\_\-Dev::PCIinterfaceFlags, npscip\_\-Dev::PCIinterfaceLen, npscip\_\-Dev::pIPinterfaceReg, npscip\_\-Dev::pPCIinterfaceReg, PSCIP\_\-DEFAULT\_\-DEV, pscip\_\-dev\_\-number, pscip\_\-devices, pscip\_\-driver, pscip\_\-major, PSCIP\_\-MAX\_\-CHAN, PSCIP\_\-MAX\_\-DEV, pscip\_\-minor, pscip\_\-procfs\_\-register(), and npscip\_\-Dev::refcount.\hypertarget{pscip_8c_cc378f45db0de3c2cc0ac3b485843a26}{
\index{pscip.c@{pscip.c}!module\_\-exit@{module\_\-exit}}
\index{module\_\-exit@{module\_\-exit}!pscip.c@{pscip.c}}
\subsubsection[{module\_\-exit}]{\setlength{\rightskip}{0pt plus 5cm}module\_\-exit (exit\_\-pscip\_\-module)}}
\label{pscip_8c_cc378f45db0de3c2cc0ac3b485843a26}


Registering last function to be called when de-inserting the module. 

\hypertarget{pscip_8c_a5f2885f1cd67ce96d2afa5a402033a8}{
\index{pscip.c@{pscip.c}!module\_\-init@{module\_\-init}}
\index{module\_\-init@{module\_\-init}!pscip.c@{pscip.c}}
\subsubsection[{module\_\-init}]{\setlength{\rightskip}{0pt plus 5cm}module\_\-init (init\_\-pscip\_\-module)}}
\label{pscip_8c_a5f2885f1cd67ce96d2afa5a402033a8}


Registering first function to be called when inserting the module. 

\hypertarget{pscip_8c_d94b36675e7eb067ea3ce6ff9e244a44}{
\index{pscip.c@{pscip.c}!MODULE\_\-LICENSE@{MODULE\_\-LICENSE}}
\index{MODULE\_\-LICENSE@{MODULE\_\-LICENSE}!pscip.c@{pscip.c}}
\subsubsection[{MODULE\_\-LICENSE}]{\setlength{\rightskip}{0pt plus 5cm}MODULE\_\-LICENSE (\char`\"{}GPL\char`\"{})}}
\label{pscip_8c_d94b36675e7eb067ea3ce6ff9e244a44}


$<$ If defined, Real Time IP enabled. 

RTAI is inharited from Electtra and not changed/looked into Determines the driver's licencing. If missing the kernel shouts about taining it \hypertarget{pscip_8c_b83342c5da927e07b71f9e9e0f2ac1ba}{
\index{pscip.c@{pscip.c}!module\_\-param@{module\_\-param}}
\index{module\_\-param@{module\_\-param}!pscip.c@{pscip.c}}
\subsubsection[{module\_\-param}]{\setlength{\rightskip}{0pt plus 5cm}module\_\-param ({\bf pscip\_\-major}, \/  int, \/  0)}}
\label{pscip_8c_b83342c5da927e07b71f9e9e0f2ac1ba}


Macro defining pscip\_\-major module parameters. 

\hypertarget{pscip_8c_eeb0a2d9412b2598de60d3ab78abd559}{
\index{pscip.c@{pscip.c}!module\_\-param@{module\_\-param}}
\index{module\_\-param@{module\_\-param}!pscip.c@{pscip.c}}
\subsubsection[{module\_\-param}]{\setlength{\rightskip}{0pt plus 5cm}module\_\-param ({\bf hack\_\-driver}, \/  int, \/  0)}}
\label{pscip_8c_eeb0a2d9412b2598de60d3ab78abd559}


Macro defining hack\_\-driver module parameters. 

\hypertarget{pscip_8c_905ed87697ca0a454ff57d69ec22af48}{
\index{pscip.c@{pscip.c}!MODULE\_\-PARM\_\-DESC@{MODULE\_\-PARM\_\-DESC}}
\index{MODULE\_\-PARM\_\-DESC@{MODULE\_\-PARM\_\-DESC}!pscip.c@{pscip.c}}
\subsubsection[{MODULE\_\-PARM\_\-DESC}]{\setlength{\rightskip}{0pt plus 5cm}MODULE\_\-PARM\_\-DESC ({\bf hack\_\-driver}, \/  \char`\"{}PSCIP special functions for developers:$\backslash$n -- no func implemented at the moment --\char`\"{})}}
\label{pscip_8c_905ed87697ca0a454ff57d69ec22af48}


\hypertarget{pscip_8c_4c7ea419630b353c78ac1aaf9cdd3a78}{
\index{pscip.c@{pscip.c}!MODULE\_\-PARM\_\-DESC@{MODULE\_\-PARM\_\-DESC}}
\index{MODULE\_\-PARM\_\-DESC@{MODULE\_\-PARM\_\-DESC}!pscip.c@{pscip.c}}
\subsubsection[{MODULE\_\-PARM\_\-DESC}]{\setlength{\rightskip}{0pt plus 5cm}MODULE\_\-PARM\_\-DESC ({\bf pscip\_\-dev\_\-number}, \/  \char`\"{}PSCIP number of devices which can be run using this driver\char`\"{})}}
\label{pscip_8c_4c7ea419630b353c78ac1aaf9cdd3a78}


\hypertarget{pscip_8c_486932d17bb9779ad2d987b610b1ab88}{
\index{pscip.c@{pscip.c}!MODULE\_\-PARM\_\-DESC@{MODULE\_\-PARM\_\-DESC}}
\index{MODULE\_\-PARM\_\-DESC@{MODULE\_\-PARM\_\-DESC}!pscip.c@{pscip.c}}
\subsubsection[{MODULE\_\-PARM\_\-DESC}]{\setlength{\rightskip}{0pt plus 5cm}MODULE\_\-PARM\_\-DESC ({\bf pscip\_\-major}, \/  \char`\"{}PSCIP major  {\em number}, \/  if defined by user\char`\"{})}}
\label{pscip_8c_486932d17bb9779ad2d987b610b1ab88}


\hypertarget{pscip_8c_1f2bd07aac6fb1d9a27f635dac28669b}{
\index{pscip.c@{pscip.c}!probe@{probe}}
\index{probe@{probe}!pscip.c@{pscip.c}}
\subsubsection[{probe}]{\setlength{\rightskip}{0pt plus 5cm}static int probe (struct pci\_\-dev $\ast$ {\em dev}, \/  const struct pci\_\-device\_\-id $\ast$ {\em id})\hspace{0.3cm}{\tt  \mbox{[}static\mbox{]}}}}
\label{pscip_8c_1f2bd07aac6fb1d9a27f635dac28669b}


Function called when the device (Carrier card) is detected on the PCI bus. 

It performs most of the initialization for both IPs on the carrier and registers the character device in the kernel.

\begin{Desc}
\item[Parameters:]
\begin{description}
\item[{\em $\ast$dev}]- PCI device structure representing PCI device \item[{\em $\ast$id}]- \end{description}
\end{Desc}


Definition at line 1202 of file pscip.c.

References npscip\_\-Dev::carrier\_\-number, npscip\_\-Dev::cdev, checkmem\_\-device(), FALSE, init\_\-pscip(), npscip\_\-Dev::IP\_\-in\_\-slot, npscip\_\-Dev::IP\_\-IOspaceAddr, npscip\_\-Dev::IP\_\-IOspaceFlags, npscip\_\-Dev::IP\_\-IOspaceLen, npscip\_\-Dev::IPinterfaceAddr, npscip\_\-Dev::IPinterfaceFlags, npscip\_\-Dev::IPinterfaceLen, npscip\_\-Dev::IPnumber, npscip\_\-Dev::irq, MAX\_\-NUMER\_\-OF\_\-IPs\_\-ON\_\-CARRIER, module\_\-name, num\_\-pscip, npscip\_\-Dev::pciaddr, npscip\_\-Dev::pcidev, npscip\_\-Dev::PCIinterfaceAddr, npscip\_\-Dev::PCIinterfaceFlags, npscip\_\-Dev::PCIinterfaceLen, npscip\_\-Dev::pIP\_\-IOspace, npscip\_\-Dev::pIPinterfaceReg, npscip\_\-Dev::pPCIinterfaceReg, PSCIP\_\-CHAN\_\-OFFSET, pscip\_\-dev\_\-number, pscip\_\-devices, pscip\_\-fops, pscip\_\-get\_\-PCI\_\-info(), PSCIP\_\-IP\_\-OFFSET, pscip\_\-major, pscip\_\-minor, npscip\_\-Dev::pscip\_\-number, and TRUE.\hypertarget{pscip_8c_893e661d3b98b489ec148f53552157ef}{
\index{pscip.c@{pscip.c}!pscip\_\-assign\_\-irq@{pscip\_\-assign\_\-irq}}
\index{pscip\_\-assign\_\-irq@{pscip\_\-assign\_\-irq}!pscip.c@{pscip.c}}
\subsubsection[{pscip\_\-assign\_\-irq}]{\setlength{\rightskip}{0pt plus 5cm}int pscip\_\-assign\_\-irq ({\bf pscip\_\-Dev} {\em dev})}}
\label{pscip_8c_893e661d3b98b489ec148f53552157ef}


Function registering interrupt in the cernel. 

Calles function which registers interrupt handler.

\begin{Desc}
\item[Parameters:]
\begin{description}
\item[{\em dev}]- device structure \end{description}
\end{Desc}


Definition at line 844 of file pscip.c.

References npscip\_\-Dev::irq, and pscip\_\-irq\_\-handler().

Referenced by pscip\_\-open().\hypertarget{pscip_8c_189aa0b40518a5c712bb9c24f4da5e7f}{
\index{pscip.c@{pscip.c}!pscip\_\-ioctl@{pscip\_\-ioctl}}
\index{pscip\_\-ioctl@{pscip\_\-ioctl}!pscip.c@{pscip.c}}
\subsubsection[{pscip\_\-ioctl}]{\setlength{\rightskip}{0pt plus 5cm}static int pscip\_\-ioctl (struct inode $\ast$ {\em inode}, \/  struct file $\ast$ {\em filp}, \/  unsigned int {\em cmd}, \/  unsigned long {\em arg})\hspace{0.3cm}{\tt  \mbox{[}static\mbox{]}}}}
\label{pscip_8c_189aa0b40518a5c712bb9c24f4da5e7f}


Function implementing ioctl control. 

Used to read-write I/O operations and IP configuration.

\begin{Desc}
\item[Parameters:]
\begin{description}
\item[{\em $\ast$inode}]- contains cdev structure, allows to get our device structure \item[{\em $\ast$filp}]- pointer to a file structure  cmd - command to be executed \item[{\em arg}]- data passed from/to kernel space \end{description}
\end{Desc}


Definition at line 302 of file pscip.c.

References pscip\_\-t::address, pscip\_\-t::chan, id\_\-pscip, IP\_\-DEBUG\_\-read\_\-FPGA\_\-stat(), IP\_\-DEBUG\_\-read\_\-IP\_\-interface\_\-reg(), IP\_\-read\_\-byte(), IP\_\-write\_\-byte(), k\_\-pscip\_\-clear\_\-err\_\-counter(), k\_\-pscip\_\-dump\_\-all\_\-registers(), k\_\-pscip\_\-rdfpgastat(), k\_\-pscip\_\-rdlinkstat(), k\_\-pscip\_\-rdstatistics(), k\_\-pscip\_\-rdwave(), k\_\-pscip\_\-read(), k\_\-pscip\_\-reset(), k\_\-pscip\_\-wrhiprio(), k\_\-pscip\_\-write(), k\_\-pscip\_\-wrwave(), npscip\_\-Dev::pciaddr, npscip\_\-Dev::pIP\_\-IOspace, npscip\_\-Dev::pIPinterfaceReg, PSCIP\_\-CLRCOUNTER, PSCIP\_\-DUMPFPGAREGS, PSCIP\_\-FPGA\_\-REG, PSCIP\_\-PCIADDR, PSCIP\_\-RDFPGASTAT, PSCIP\_\-RDLINKSTAT, PSCIP\_\-RDSTATISTICS, PSCIP\_\-RDWAVE, PSCIP\_\-READ, PSCIP\_\-READIPREG, PSCIP\_\-RESET, PSCIP\_\-WRHIPRIO, PSCIP\_\-WRITE, PSCIP\_\-WRITEIPREG, PSCIP\_\-WRWAVE, npscip\_\-Dev::sem, pscip\_\-t::stat, and val.\hypertarget{pscip_8c_fbfd4fc2195d42c9aaed821f40c33e12}{
\index{pscip.c@{pscip.c}!pscip\_\-irq\_\-bh0@{pscip\_\-irq\_\-bh0}}
\index{pscip\_\-irq\_\-bh0@{pscip\_\-irq\_\-bh0}!pscip.c@{pscip.c}}
\subsubsection[{pscip\_\-irq\_\-bh0}]{\setlength{\rightskip}{0pt plus 5cm}void pscip\_\-irq\_\-bh0 (unsigned long {\em data})}}
\label{pscip_8c_fbfd4fc2195d42c9aaed821f40c33e12}


Tasklet used to send a SIGIO to user space, chan 0. 

pscip\_\-irq\_\-bh0 - called for interrupts on channel 0 of IP tasklets are a special function that may be scheduled to run, in software interrupt context, at a system-determined safe time.

They may be scheduled to run multiple times, but tasklet scheduling is not cumulative; the tasklet runs only once, even if it is requested repeatedly before it is launched. NOTE: tasklet function should return irqreturn\_\-t but here this functions are initialized in \char`\"{}unconventional\char`\"{} way and they need to return void.....

\begin{Desc}
\item[Parameters:]
\begin{description}
\item[{\em data}]- data to be passed to the tasklet - anything cased to unsigned long - probably pointer \end{description}
\end{Desc}


Definition at line 686 of file pscip.c.

References npscip\_\-Dev::lock, npscip\_\-Dev::queue, and queue\_\-flag.

Referenced by init\_\-pscip().\hypertarget{pscip_8c_0792694f70567c97210fbbee0e0702a3}{
\index{pscip.c@{pscip.c}!pscip\_\-irq\_\-bh1@{pscip\_\-irq\_\-bh1}}
\index{pscip\_\-irq\_\-bh1@{pscip\_\-irq\_\-bh1}!pscip.c@{pscip.c}}
\subsubsection[{pscip\_\-irq\_\-bh1}]{\setlength{\rightskip}{0pt plus 5cm}void pscip\_\-irq\_\-bh1 (unsigned long {\em data})}}
\label{pscip_8c_0792694f70567c97210fbbee0e0702a3}


Tasklet used to send a SIGIO to user space, chan 0. 

pscip\_\-irq\_\-bh1 - called for interrupts on channel 1 of IP tasklets are a special function that may be scheduled to run, in software interrupt context, at a system-determined safe time.

They may be scheduled to run multiple times, but tasklet scheduling is not cumulative; the tasklet runs only once, even if it is requested repeatedly before it is launched. NOTE: tasklet function should return irqreturn\_\-t but here this functions are initialized in \char`\"{}unconventional\char`\"{} way and they need to return void.....

\begin{Desc}
\item[Parameters:]
\begin{description}
\item[{\em data}]- data to be passed to the tasklet - anything cased to unsigned long - probably pointer \end{description}
\end{Desc}


Definition at line 722 of file pscip.c.

References npscip\_\-Dev::lock, npscip\_\-Dev::queue, and queue\_\-flag.

Referenced by init\_\-pscip().\hypertarget{pscip_8c_0356476f23b4337ef94d580026be8bbb}{
\index{pscip.c@{pscip.c}!pscip\_\-irq\_\-handler@{pscip\_\-irq\_\-handler}}
\index{pscip\_\-irq\_\-handler@{pscip\_\-irq\_\-handler}!pscip.c@{pscip.c}}
\subsubsection[{pscip\_\-irq\_\-handler}]{\setlength{\rightskip}{0pt plus 5cm}irqreturn\_\-t pscip\_\-irq\_\-handler (int {\em pciirq}, \/  void $\ast$ {\em dev\_\-id}, \/  struct pt\_\-regs $\ast$ {\em regs})}}
\label{pscip_8c_0356476f23b4337ef94d580026be8bbb}


Interrupt hanlder. 

Called when an interrupt occures.

It recognize whether the interrupt should be handled, checks errors and schedules appropraite tasklet function. Irq handler immediately reset pending interrupts to avoid CPU freezing

\begin{Desc}
\item[Parameters:]
\begin{description}
\item[{\em pciirq}]- interrupt number registered for this device,hardware-wise number of interrupt line \item[{\em $\ast$dev\_\-id}]- a sort of client data \item[{\em $\ast$regs}]- holds a snapshot of the processor�s context before the processor entered interrupt code \end{description}
\end{Desc}


Definition at line 761 of file pscip.c.

References pscip\_\-stats\_\-t::err\_\-flag, IP\_\-read\_\-word(), IP\_\-write\_\-word(), npscip\_\-Dev::irqflag, k\_\-if\_\-handle\_\-irq(), pscip\_\-stats\_\-t::link\_\-err\_\-flag, pscip\_\-stats\_\-t::linkdown\_\-cnt, npscip\_\-Dev::pIP\_\-IOspace, PLX9030\_\-INTCSR\_\-LINTI1\_\-STAT, npscip\_\-Dev::pPCIinterfaceReg, PSCIP\_\-ERRCLEAR\_\-MSK, PSCIP\_\-ERROR\_\-MSK, PSCIP\_\-FPGA\_\-HIGH, PSCIP\_\-FPGA\_\-LOW, PSCIP\_\-FPGA\_\-MID, PSCIP\_\-IRQ\_\-SERVED, PSCIP\_\-LINKDOWN\_\-MSK, PSCIP\_\-RDWAVE\_\-IDX, PSCIP\_\-READ\_\-IDX, PSCIP\_\-REG\_\-STAT, npscip\_\-Dev::pscip\_\-tsk, PSCIP\_\-TXERR\_\-MSK, PSCIP\_\-WRHPRIO\_\-IDX, PSCIP\_\-WRITE\_\-IDX, PSCIP\_\-WRWAVE\_\-IDX, npscip\_\-Dev::stats, and pscip\_\-stats\_\-t::txerr\_\-cnt.

Referenced by pscip\_\-assign\_\-irq().\hypertarget{pscip_8c_38ed0a47b49920ca3b02392764122bed}{
\index{pscip.c@{pscip.c}!pscip\_\-open@{pscip\_\-open}}
\index{pscip\_\-open@{pscip\_\-open}!pscip.c@{pscip.c}}
\subsubsection[{pscip\_\-open}]{\setlength{\rightskip}{0pt plus 5cm}static int pscip\_\-open (struct inode $\ast$ {\em inode}, \/  struct file $\ast$ {\em filp})\hspace{0.3cm}{\tt  \mbox{[}static\mbox{]}}}}
\label{pscip_8c_38ed0a47b49920ca3b02392764122bed}


Function called when the node represeting driver is opened. 

Does any initialization for later operations.

While creating first reference (first file/node open) interrupt on give IP are enabled

\begin{Desc}
\item[Parameters:]
\begin{description}
\item[{\em $\ast$inode}]- contains cdev structure, allows to get our device structure \item[{\em $\ast$filp}]pointer to a file structure \end{description}
\end{Desc}


Error code. Store error code in case of errro 

Definition at line 200 of file pscip.c.

References npscip\_\-Dev::irq, k\_\-irq\_\-enable(), pscip\_\-assign\_\-irq(), npscip\_\-Dev::refcount, and npscip\_\-Dev::sem.\hypertarget{pscip_8c_4c7cd65a5fb7a2c6dc317b41df5622dd}{
\index{pscip.c@{pscip.c}!pscip\_\-procfs\_\-register@{pscip\_\-procfs\_\-register}}
\index{pscip\_\-procfs\_\-register@{pscip\_\-procfs\_\-register}!pscip.c@{pscip.c}}
\subsubsection[{pscip\_\-procfs\_\-register}]{\setlength{\rightskip}{0pt plus 5cm}static int pscip\_\-procfs\_\-register (void)\hspace{0.3cm}{\tt  \mbox{[}static\mbox{]}}}}
\label{pscip_8c_4c7cd65a5fb7a2c6dc317b41df5622dd}


Register proc filesystem entries. 

It creates the appropriate directory in /proc/driver/ and fills it with nodes representing each IP It registers functions implemented for proc\_\-fs interface 

Definition at line 1058 of file pscip.c.

References npscip\_\-Proc::count, module\_\-name, proc\_\-model\_\-dir, pscip\_\-dev\_\-number, pscip\_\-read\_\-proc(), and pscip\_\-write\_\-proc().

Referenced by init\_\-pscip\_\-module().\hypertarget{pscip_8c_8675bb66851e6553e8d99ec533e2f6ec}{
\index{pscip.c@{pscip.c}!pscip\_\-read\_\-proc@{pscip\_\-read\_\-proc}}
\index{pscip\_\-read\_\-proc@{pscip\_\-read\_\-proc}!pscip.c@{pscip.c}}
\subsubsection[{pscip\_\-read\_\-proc}]{\setlength{\rightskip}{0pt plus 5cm}static int pscip\_\-read\_\-proc (char $\ast$ {\em page}, \/  char $\ast$$\ast$ {\em start}, \/  off\_\-t {\em offset}, \/  int {\em count}, \/  int $\ast$ {\em eof}, \/  void $\ast$ {\em data})\hspace{0.3cm}{\tt  \mbox{[}static\mbox{]}}}}
\label{pscip_8c_8675bb66851e6553e8d99ec533e2f6ec}


Create a proc filesystem read entry. 

\begin{Desc}
\item[Parameters:]
\begin{description}
\item[{\em $\ast$page}]- page pointer, buffer to which date is writen \item[{\em $\ast$$\ast$start}]- used by the function to say where the interesting data has been written \item[{\em offset}]- the same as count \item[{\em count}]- \item[{\em $\ast$eof}]- points to an integer that must be set by the driver to signal that it has no more data to return \item[{\em data}]- \end{description}
\end{Desc}


Definition at line 879 of file pscip.c.

References npscip\_\-Dev::carrier\_\-number, pscip\_\-fpga\_\-t::channel, pscip\_\-stats\_\-t::cntrl\_\-stat, npscip\_\-Proc::count, pscip\_\-fpga\_\-t::high\_\-data, id\_\-pscip, npscip\_\-Dev::IP\_\-in\_\-slot, npscip\_\-Dev::IPnumber, k\_\-pscip\_\-rdfpgastat(), pscip\_\-stats\_\-t::linkdown\_\-cnt, pscip\_\-devices, PSCIP\_\-DSP\_\-MSK, PSCIP\_\-INPUTBUF\_\-MSK, PSCIP\_\-LINKDOWN\_\-MSK, PSCIP\_\-MAX\_\-CHAN, PSCIP\_\-REMLOC\_\-MSK, PSCIP\_\-TXERR\_\-MSK, pscip\_\-stats\_\-t::rx, npscip\_\-Dev::sem, npscip\_\-Dev::stats, pscip\_\-stats\_\-t::tx, pscip\_\-stats\_\-t::txerr\_\-cnt, and ver.

Referenced by pscip\_\-procfs\_\-register().\hypertarget{pscip_8c_ac05b05bbeed23bd53e662d75d14d4f8}{
\index{pscip.c@{pscip.c}!pscip\_\-release@{pscip\_\-release}}
\index{pscip\_\-release@{pscip\_\-release}!pscip.c@{pscip.c}}
\subsubsection[{pscip\_\-release}]{\setlength{\rightskip}{0pt plus 5cm}static int pscip\_\-release (struct inode $\ast$ {\em inode}, \/  struct file $\ast$ {\em filp})\hspace{0.3cm}{\tt  \mbox{[}static\mbox{]}}}}
\label{pscip_8c_ac05b05bbeed23bd53e662d75d14d4f8}


Function called when the node represeting driver is closed. 

Invoked when the file structure is being released.

If the last reference is closed, interrupts are disabled on give IP represented by the file/node

\begin{Desc}
\item[Parameters:]
\begin{description}
\item[{\em $\ast$inode}]- contains cdev structure, allows to get our device structure \item[{\em $\ast$filp}]pointer to a file structure \end{description}
\end{Desc}


Definition at line 250 of file pscip.c.

References npscip\_\-Dev::irq, k\_\-irq\_\-disable(), npscip\_\-Dev::refcount, and npscip\_\-Dev::sem.\hypertarget{pscip_8c_bf1022c0b7beb0fa86c449b1ba63d58b}{
\index{pscip.c@{pscip.c}!pscip\_\-write\_\-proc@{pscip\_\-write\_\-proc}}
\index{pscip\_\-write\_\-proc@{pscip\_\-write\_\-proc}!pscip.c@{pscip.c}}
\subsubsection[{pscip\_\-write\_\-proc}]{\setlength{\rightskip}{0pt plus 5cm}static int pscip\_\-write\_\-proc (struct file $\ast$ {\em filp}, \/  const char $\ast$ {\em buffer}, \/  unsigned long {\em count}, \/  void $\ast$ {\em data})\hspace{0.3cm}{\tt  \mbox{[}static\mbox{]}}}}
\label{pscip_8c_bf1022c0b7beb0fa86c449b1ba63d58b}


creat a proc file system write entry 

\begin{Desc}
\item[Parameters:]
\begin{description}
\item[{\em filep}]- file structure pointer \item[{\em $\ast$buffer}]- data from the user \item[{\em count}]- \item[{\em $\ast$data}]- \end{description}
\end{Desc}


Definition at line 964 of file pscip.c.

References pscip\_\-t::address, pscip\_\-t::chan, npscip\_\-Proc::count, pscip\_\-t::data, id\_\-pscip, k\_\-pscip\_\-reset(), k\_\-pscip\_\-wrhiprio(), k\_\-pscip\_\-write(), k\_\-pscip\_\-wrwave(), pscip\_\-devices, PSCIP\_\-PROC\_\-HPRIO, PSCIP\_\-PROC\_\-RESET, PSCIP\_\-PROC\_\-WRITE, PSCIP\_\-PROC\_\-WRWAVE, npscip\_\-Dev::sem, pscip\_\-t::stat, and val.

Referenced by pscip\_\-procfs\_\-register().\hypertarget{pscip_8c_fdbceb4ede4f982e2e68f2ee7c0cf6e1}{
\index{pscip.c@{pscip.c}!remove@{remove}}
\index{remove@{remove}!pscip.c@{pscip.c}}
\subsubsection[{remove}]{\setlength{\rightskip}{0pt plus 5cm}static void remove (struct pci\_\-dev $\ast$ {\em pcidev})\hspace{0.3cm}{\tt  \mbox{[}static\mbox{]}}}}
\label{pscip_8c_fdbceb4ede4f982e2e68f2ee7c0cf6e1}


Function called when the device (Carrier) is removed from the PCI bus. 

Does all the unregister/remove/free stuff in reverse order than probe.

\begin{Desc}
\item[Parameters:]
\begin{description}
\item[{\em $\ast$pcidev}]- structure representing PCI device \end{description}
\end{Desc}


Definition at line 1393 of file pscip.c.

References npscip\_\-Dev::pIP\_\-IOspace, npscip\_\-Dev::pIPinterfaceReg, npscip\_\-Dev::pPCIinterfaceReg, pscip\_\-devices, and npscip\_\-Dev::pscip\_\-number.

\subsection{Variable Documentation}
\hypertarget{pscip_8c_1aa2a01bfd35c87921617d64a164bda7}{
\index{pscip.c@{pscip.c}!hack\_\-driver@{hack\_\-driver}}
\index{hack\_\-driver@{hack\_\-driver}!pscip.c@{pscip.c}}
\subsubsection[{hack\_\-driver}]{\setlength{\rightskip}{0pt plus 5cm}int {\bf hack\_\-driver} = 0\hspace{0.3cm}{\tt  \mbox{[}static\mbox{]}}}}
\label{pscip_8c_1aa2a01bfd35c87921617d64a164bda7}


Module paramter - options for developer. 

At the moment ot used 

Definition at line 113 of file pscip.c.\hypertarget{pscip_8c_4c2d9327de87c6be0b54f793adc7f1fb}{
\index{pscip.c@{pscip.c}!module\_\-name@{module\_\-name}}
\index{module\_\-name@{module\_\-name}!pscip.c@{pscip.c}}
\subsubsection[{module\_\-name}]{\setlength{\rightskip}{0pt plus 5cm}const char$\ast$ {\bf module\_\-name} = \char`\"{}pscip\char`\"{}}}
\label{pscip_8c_4c2d9327de87c6be0b54f793adc7f1fb}


module name. 



Definition at line 109 of file pscip.c.

Referenced by exit\_\-pscip\_\-module(), init\_\-module(), probe(), and pscip\_\-procfs\_\-register().\hypertarget{pscip_8c_e88af066b7442e9fc6cf5228860c2cc5}{
\index{pscip.c@{pscip.c}!proc\_\-model\_\-dir@{proc\_\-model\_\-dir}}
\index{proc\_\-model\_\-dir@{proc\_\-model\_\-dir}!pscip.c@{pscip.c}}
\subsubsection[{proc\_\-model\_\-dir}]{\setlength{\rightskip}{0pt plus 5cm}struct proc\_\-dir\_\-entry$\ast$ {\bf proc\_\-model\_\-dir}\hspace{0.3cm}{\tt  \mbox{[}static\mbox{]}}}}
\label{pscip_8c_e88af066b7442e9fc6cf5228860c2cc5}


Structure defining entry in proc file system. 



Definition at line 186 of file pscip.c.

Referenced by exit\_\-pscip\_\-module(), and pscip\_\-procfs\_\-register().\hypertarget{pscip_8c_5227c6d12c17c33f868af17fb077dc62}{
\index{pscip.c@{pscip.c}!pscip\_\-dev\_\-number@{pscip\_\-dev\_\-number}}
\index{pscip\_\-dev\_\-number@{pscip\_\-dev\_\-number}!pscip.c@{pscip.c}}
\subsubsection[{pscip\_\-dev\_\-number}]{\setlength{\rightskip}{0pt plus 5cm}int {\bf pscip\_\-dev\_\-number} = 0\hspace{0.3cm}{\tt  \mbox{[}static\mbox{]}}}}
\label{pscip_8c_5227c6d12c17c33f868af17fb077dc62}


Module parameter - number of IPs. 

Number of devices (IP modules) which will be in the computer, if not set, the DEFAULT number will be give, is limited by the MAX\_\-DEV number 

Definition at line 112 of file pscip.c.

Referenced by exit\_\-pscip\_\-module(), init\_\-pscip\_\-module(), probe(), and pscip\_\-procfs\_\-register().\hypertarget{pscip_8c_b8e932f1cc0da35f7c08d4bcfb8afaee}{
\index{pscip.c@{pscip.c}!pscip\_\-driver@{pscip\_\-driver}}
\index{pscip\_\-driver@{pscip\_\-driver}!pscip.c@{pscip.c}}
\subsubsection[{pscip\_\-driver}]{\setlength{\rightskip}{0pt plus 5cm}struct pci\_\-driver {\bf pscip\_\-driver}\hspace{0.3cm}{\tt  \mbox{[}static\mbox{]}}}}
\label{pscip_8c_b8e932f1cc0da35f7c08d4bcfb8afaee}


\textbf{Initial value:}

\begin{Code}\begin{verbatim} {
 .name = "pscip",
 .id_table = ids,
 .probe = probe,
 .remove = remove,
}
\end{verbatim}
\end{Code}
Structure describing PCI device (IP). 

Structure consists of a number of function callbacks and variables that describe the PCI driver to the PCI core 

Definition at line 176 of file pscip.c.

Referenced by exit\_\-pscip\_\-module(), and init\_\-pscip\_\-module().\hypertarget{pscip_8c_1969c4e32e8a199df965d0e49e7bdb8e}{
\index{pscip.c@{pscip.c}!pscip\_\-fops@{pscip\_\-fops}}
\index{pscip\_\-fops@{pscip\_\-fops}!pscip.c@{pscip.c}}
\subsubsection[{pscip\_\-fops}]{\setlength{\rightskip}{0pt plus 5cm}struct file\_\-operations {\bf pscip\_\-fops}\hspace{0.3cm}{\tt  \mbox{[}static\mbox{]}}}}
\label{pscip_8c_1969c4e32e8a199df965d0e49e7bdb8e}


\textbf{Initial value:}

\begin{Code}\begin{verbatim} 
{
 .owner  =   THIS_MODULE,
 .ioctl  = pscip_ioctl,
 .open  = pscip_open,
 .release = pscip_release
 
}
\end{verbatim}
\end{Code}
File operation structure. 

Associates the driver (major number) with device operations. The structure is a collection of function pointers. Each open file is associated with its own set of functions 

Definition at line 164 of file pscip.c.

Referenced by probe().\hypertarget{pscip_8c_c42777cda72b563c2676c06d0fd98c1a}{
\index{pscip.c@{pscip.c}!pscip\_\-major@{pscip\_\-major}}
\index{pscip\_\-major@{pscip\_\-major}!pscip.c@{pscip.c}}
\subsubsection[{pscip\_\-major}]{\setlength{\rightskip}{0pt plus 5cm}int {\bf pscip\_\-major}\hspace{0.3cm}{\tt  \mbox{[}static\mbox{]}}}}
\label{pscip_8c_c42777cda72b563c2676c06d0fd98c1a}


\textbf{Initial value:}

\begin{Code}\begin{verbatim} PSCIP_MAJOR_NUMBER

 
 module_param(pscip_dev_number, int, 0)
\end{verbatim}
\end{Code}
Global variable with major numer associated with the driver. 

A default major number can be set in the header, if defalut value is 0 the major number is set dynamically unless user pass major number as a parameter 

Definition at line 114 of file pscip.c.

Referenced by exit\_\-pscip\_\-module(), init\_\-pscip\_\-module(), and probe().\hypertarget{pscip_8c_422f87b12a1ce74ba2ba50eacb4ee12e}{
\index{pscip.c@{pscip.c}!pscip\_\-minor@{pscip\_\-minor}}
\index{pscip\_\-minor@{pscip\_\-minor}!pscip.c@{pscip.c}}
\subsubsection[{pscip\_\-minor}]{\setlength{\rightskip}{0pt plus 5cm}int {\bf pscip\_\-minor} = PSCIP\_\-MINOR\_\-NUMBER}}
\label{pscip_8c_422f87b12a1ce74ba2ba50eacb4ee12e}


Global variable with first minor numer associated with the driver. 



Definition at line 106 of file pscip.c.

Referenced by exit\_\-pscip\_\-module(), init\_\-pscip\_\-module(), and probe().\hypertarget{pscip_8c_8f2d8a5dd9c85bb225fbd2edf4175095}{
\index{pscip.c@{pscip.c}!pscip\_\-procdev@{pscip\_\-procdev}}
\index{pscip\_\-procdev@{pscip\_\-procdev}!pscip.c@{pscip.c}}
\subsubsection[{pscip\_\-procdev}]{\setlength{\rightskip}{0pt plus 5cm}{\bf pscip\_\-Proc} {\bf pscip\_\-procdev}\hspace{0.3cm}{\tt  \mbox{[}static\mbox{]}}}}
\label{pscip_8c_8f2d8a5dd9c85bb225fbd2edf4175095}


Proc filesystem structure . 



Definition at line 185 of file pscip.c.\hypertarget{pscip_8c_8883057b44748f328f9ab36e5bcf2a8f}{
\index{pscip.c@{pscip.c}!queue\_\-flag@{queue\_\-flag}}
\index{queue\_\-flag@{queue\_\-flag}!pscip.c@{pscip.c}}
\subsubsection[{queue\_\-flag}]{\setlength{\rightskip}{0pt plus 5cm}int {\bf queue\_\-flag} = 0\hspace{0.3cm}{\tt  \mbox{[}static\mbox{]}}}}
\label{pscip_8c_8883057b44748f328f9ab36e5bcf2a8f}


Flag used in waitqueue mechanism. 

This is a provisional solution, probablyl needs better implementation 

Definition at line 91 of file pscip.c.

Referenced by k\_\-pscip\_\-rdwave(), k\_\-pscip\_\-read(), k\_\-pscip\_\-wrhiprio(), k\_\-pscip\_\-write(), k\_\-pscip\_\-wrwave(), pscip\_\-irq\_\-bh0(), and pscip\_\-irq\_\-bh1().\hypertarget{pscip_8c_699de8ad6de47a5f252fe943bbe146e5}{
\index{pscip.c@{pscip.c}!ver@{ver}}
\index{ver@{ver}!pscip.c@{pscip.c}}
\subsubsection[{ver}]{\setlength{\rightskip}{0pt plus 5cm}char {\bf ver}\mbox{[}10\mbox{]} = \char`\"{}1.0\char`\"{}\hspace{0.3cm}{\tt  \mbox{[}static\mbox{]}}}}
\label{pscip_8c_699de8ad6de47a5f252fe943bbe146e5}


Module/driver version. 



Definition at line 108 of file pscip.c.

Referenced by pscip\_\-read\_\-proc().